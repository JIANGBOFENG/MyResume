\resheading{推荐标的:分众传媒}
  \begin{itemize}[leftmargin=*]
    \item
      \textbf{行业:具备竞争优势的周期成长性行业}
      {\small
      \begin{itemize}
        \item 梯媒从物业手中租赁点位再出租给广告主,点位成本较为固定,经营杠杆高
        \item 优质点位资源有限,行业内企业\textbf{具有先发优势},\textbf{壁垒高},上下游面对的物业和广告主较为分散,\textbf{议价能力强}
        \item 广告行业受宏观经济影响具有\textbf{周期性},梯媒属于品牌广告,周期性高于广告行业整体
        \item\textbf{ 广告行业长期增速高于GDP增速},因相比其他广告渠道性价比高,\textbf{梯媒在广告行业内份额不断扩大}
      \end{itemize}
      }
    \item
      \textbf{基本面:地位稳固的行业寡头}
      {\small
      \begin{itemize}
        \item 分众\textbf{占据一二线城市的核心点位},面向人群消费能力强,不断扩充点位巩固市场,\textbf{市占率高达70\%}
        \item 分众点位质量高、触达人群广,管理层销售能力强,对点位标签化管理、打包销售,\textbf{吸引大广告主,获得溢价}
        \item 行业竞争格局固化,\textbf{分众垄断地位无可挑战},新挑战者无空间立足,\textbf{点位租金成本未来增幅有限}
      \end{itemize}
      }
      \item
      \textbf{核心逻辑:价格战重塑梯媒行业}
      {\small
      \begin{itemize}
        \item 价格战使行业竞争格局固化,\textbf{分众垄断地位无可挑战},新潮无力竞争,行业潜在进入者也无足够空间立足,投资安全性有足够保障
        \item 价格战使行业出现投资机遇,\textbf{分众点位扩充了75\%},\textbf{刊例价下降了55\%},梯媒渠道体现出较高性价比,为未来业绩创造了高弹性的空间
      \end{itemize}
      }
    \item
      \textbf{短期逻辑:景气周期提价带来业绩确定性}
      {\small
      \begin{itemize}
        \item 宏观经济高景气,下半年或逐步呈现下行压力但幅度不大,政策选择最差温和去杠杆,不会重复18年景象,\textbf{经济周期影响业绩的风险较小}
        \item 注册制带动一级市场繁荣,创业公司增加广告投放支撑短期业绩
        \item 因刊挂率较高、刊例价较低,\textbf{2020、2021年均提价10\%},\textbf{目前看未来几年提价趋势可以延续}
      \end{itemize}
      }
    \item
      \textbf{长期逻辑:量价齐升带动业绩长期向好}
      {\small
      \begin{itemize}
        \item 价格战使刊例价大幅下降,只有历史最好水平的50\%,相比其他广告渠道性价比极高,2020年仍能逆势增长,\textbf{随着梯媒商业价值的逐步回归,刊例价有1倍的增长空间}
        \item \textbf{17到19年中国电梯保有量增速在13\%左右},随着老旧小区加装电梯的政策支持,未来增速有望进一步走高,为分众点位扩张创造空间
        \item 梯媒随着商业价值回归、电梯数量增长和渗透率提升,能够实现\textbf{量价齐升},预计2026年行业规模高于500亿
        \item 目前分众谨慎尝试下沉三四线市场,长期看分众有能力逐渐\textbf{抢占小公司的优质点位和市场份额},进一步提高自己的市场占有率和盈利空间
      \end{itemize}
      }
      \item
      \textbf{估值:}
      {\small
      \begin{itemize}
        \item 分众2021年考核目标梯媒收入130-150亿、影媒收入30亿,有较大概率实现,假设梯媒净利润率42-45\%,影媒净利率25\%,预估净利润62-75亿,动态市盈率20-24倍(按2021年3月12日收盘市值1491亿测算)
        \item 按照量价齐升的长期增长逻辑测算,未来5年分众的营收和净利润年均增长率在20\%以上,体现出不错的安全边际
      \end{itemize}
      }
      \textbf{风险:}
      {\small
      \begin{itemize}
        \item 
      \end{itemize}
      }
  \end{itemize}