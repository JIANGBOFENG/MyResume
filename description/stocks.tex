\resheading{投资历史}
  \begin{itemize}[leftmargin=*]
    \item
      \ressubsingleline{长安汽车}{汽车整车}{2019.11.19 -- 2020.11.05}
      {\small
      \begin{itemize}
        \item 基本面:行业整体下滑,合资品牌销量塌陷,企业陷入亏损
        \item 买入逻辑:估值低,经营杠杆高,新产品周期支撑短期业绩,行业反转支撑长期业绩,风险收益比高
        \item 预判可能利空:行业持续低迷;国企盈利意愿低;合资品牌持续亏损
        \item 投资总结:\textbf{持有期间股价上涨108\%,期初投资逻辑得到印证},国企盈利问题限制股价上涨空间,达到目标价清仓
      \end{itemize}
      }
    \item
      \ressubsingleline{小熊电器}{小家电}{2020.02.25 -- 2020.04.28}
      {\small
      \begin{itemize}
        \item 基本面:小家电行业景气度高,公司经营历史良好,爆款研发与营销能力强
        \item 买入逻辑:估值低,成长高,产品线上销售,疫情下宅经济受益
        \item 预判可能利空:行业竞争升级
        \item 投资总结:\textbf{持有期间股价上涨49\%,期初投资逻辑得到印证},因一季报不及预期、对估值比较保守清仓
      \end{itemize}
      }
    \item
      \ressubsingleline{东阿阿胶}{中医保健品}{2020.04.03 -- 2021.01.20}
      {\small
      \begin{itemize}
        \item 基本面:阿胶市场稳步增长,公司品牌溢价高,但公司连年提价并向渠道压货,导致销售不畅,存货应收双高
        \item 买入逻辑:公司管理层换届暂停提价策略,控制发货清理渠道库存,预计在2021年之前库存可以基本出清,大概率困境反转
        \item 预判可能利空:公司长期对消费者培育能力减弱;渠道去库存不及预期
        \item 投资总结:\textbf{持有期间股价上涨39\%,期初投资逻辑未得到印证},因新管理层想法过多而清仓
      \end{itemize}
      }
    \item
      \ressubsingleline{分众传媒}{梯媒}{2020.04.28 至今}
      {\small
      \begin{itemize}
        \item 基本面:电梯广告行业波动增长,公司因价格战、经济下行和疫情影响收入下滑成本上升,陷入经营困境
        \item 买入逻辑:公司19年开始优化点位表明新潮等竞争者对分众已不构成威胁,成本下降,收入受经济复苏和疫情消退影响逐步上升,进入复苏周期
        \item 预判可能利空:经济下行;物业公司抢占市场
        \item 投资总结:\textbf{持有期间股价上涨143\%,期初投资逻辑得到印证},公司2021年业绩向好,继续持有
      \end{itemize}
      }
      \item
      \ressubsingleline{隆基股份}{光伏}{2020.06.29 -- 2021.11.06}
      {\small
      \begin{itemize}
        \item 基本面:光伏行业进入高景气周期,公司精细化管理且行业经验领先,成本领先,前向一体化导致行业议价能力强,管理层战略眼光与执行力优秀
        \item 买入逻辑:估值低;平价上网拓宽了行业成长空间,减弱了行业周期性,公司具备最强势的行业地位,能够最大程度享受到行业红利
        \item 预判可能利空:原材料价格上涨影响行业规模扩张;新技术路径改变行业格局
        \item 投资总结:\textbf{持有期间股价上涨95\%,期初投资逻辑得到印证},行业波动大个人投资者难以持续跟踪,公司短期在行业内利润分配中处于不利地位,清仓
      \end{itemize}
      }
  \end{itemize}